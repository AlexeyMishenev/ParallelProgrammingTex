{ %section7_1
	\subsection{Порядок выполнения работы}
	\begin{enumerate}
		\itemИспользуя любую утилиту для анализа статистики кэш-промахов (например, valgrind, как на лекции), изобразить на графике зависимость доли кэш-промахов от числа N (при обращении как к инструкциям, так и к данным). Диапазон изменения N выбрать самостоятельно. 
		\itemЗадание на "четвёрку": обосновать выбор диапазона N в экспериментах из п.1 (используя теоретические соображения и/или результаты экспериментов) и объяснить характер полученной зависимости доли кэш-промахов от N.
		\itemЗадание на "пятёрку": при обнаружении проблем с совместным использованием кэша несколькими потоками локализовать источник проблемы и внести правки для устранения проблемы, подтвердив эффективность предложенного решения результатами экспериментов; при отсутствии проблем, обоснованно доказать, почему сделан такой вывод.
		\itemПровести эксперименты для самостоятельно выбранного диапазона N, используя в экспериментах функцию srand(const), где const не меняется между итерациями цикла верхнего уровня. Результаты замеров времени каждого из ста прогонов должны сохраняться в файл (сохранение в файл должно быть реализовано вне области кода, для которой проводятся замеры).
		\itemПолученные замеры времени использовать для расчёта доверительного интервала суммарного времени выполнения всех этапов программы с доверительной вероятностью 0.95. Для этого нужно для каждой итерации внешнего цикла просуммировать длительности всех этапов, а потом использовать накопленную сумму при расчёте доверительного интервала. На графике изобразить зависимость ширины доверительного интервала (относительную и абсолютную) от N. 
		\itemЗадание на "пятёрку": для нескольких разных N изобразить на графике зависимость ширины доверительного интервала от используемого в расчётах количества замеров (совпадающего с количеством итераций внешнего уровня). Для этого не нужно проводить новые эксперименты, т.к. можно использовать  полученные в п.4 результаты, постепенно уменьшая количество используемых в расчётах замеров k от ста до двух.
		\itemЗадание на "пятёрку": для каждого N выявить минимальное количество замеров $k_{min}$, необходимое для получения 95\%-го доверительного интервала, ширина которого не превышает 1\% от  середины доверительного интервала. Построить соответствующую зависимость $k_{min}\;=\;f(N)$ на графике.
		\itemЗаписать "прикидочное" выражение для времени выполнения программы $T(N)$ и каждого её этапа $Ti(N)$, считая, что все виды элементарных операций (все арифметические, логические, проверка условий, присваивание и т.п.) выполняются за константное время С. Записать выражение для вычислительной сложности каждого этапа и всей программы в целом в обозначениях $O$, $\Theta$, $\Omega$ от числа N. 
		\itemЗадание на "пятёрку": записать все выражения из п.8 для параллельной версии программы, считая, что в расчётах участвует ровно M вычислителей (ядер) и доля нераспараллеливаемых операций равна 0.
	\end{enumerate}
}