{ %section4_4
	\subsection{Варианты заданий}
	\Large\parВариант задания выбирается в соответствии с Таблицей. Порядок вычислений должен быть следующим:
	\begin{enumerate}
		\itemСформировать матрицу М1 размерностью N x N, заполнив её случайными вещественными числами, имеющими равномерный закон распределения в диапазоне от А до В (включительно).
		\itemИз получившейся матрицы М1 нужно сформировать матрицу М2 (размерностью N x N) в соответствии с параметром С вашего варианта. 
		\itemСформировать матрицу М3 (размерностью N x N), выполнив матричное умножение М3 = М1 х М2. 
		\itemСформировать вектор V размерностью N х 1, используя задание, указанное параметром D. 
		\itemПолученный вектор V необходимо отсортировать методом, указанным в параметре Е (для этого нельзя использовать библиотечные функции).
	\end{enumerate}
	\parПример таблицы с параметрами задания для одного студента:
	\begin{center}
		\begin{tabular}{|c|c|c|c|c|c|}
		\hline
		\textbf{ФИО студента} & \textbf{A} & \textbf{B} & \textbf{C} & \textbf{D} & \textbf{E} \\
		\hline
		Иванов Иван Иванович  & -5         & 8          & 3          & 1          & 11         \\
		\hline
		\end{tabular}
	\end{center}
	\parЗначения параметров ''А'' и ''В'' задают численное значение левой и правой границы случайных величин, генерируемых в ходе выполнения лабораторной работы. 	Параметр ''С'', определяющий метод получения матрицы М2, принимает значения от одного до семи, означающие:
	\begin{enumerate}
		\itemУмножение М1 на скаляр 5.
		\itemВычитание из матрицы М1 матрицы Е размером N x N.
		\itemТранспонирование матрицы М1.
		\itemИнвертирование знака матрицы М1.
		\itemКаждый элемент М2 равен синусу симметричного элемента в М1.
		\itemКаждый элемент М2 равен десятичному логарифму модуля симметричного элемента в М1.
		\itemКаждый элемент М2 равен элементу M1, умноженному на Х, где:
			\begin{itemize}
				\item X равен числу $\pi$(3,14159265358...), если сумма номера строки и номера столбца нечётная.
				\item X равен числу $e$(2,718281828...), если сумма номера строки и номера столбца чётная.
			\end{itemize}
	\end{enumerate}
	\parПараметр ''D'', определяющий способ формирования вектора V с результатами, принимает значения от одного до девяти, означающие:
	\begin{enumerate}
		\itemМатематическое ожидание каждой строки.
		\itemВыборочная дисперсия каждой строки.
		\itemКоэффициент вариации положительных элементов каждой строки.
		\itemВыборочное среднеквадратичное отклонение каждой строки.
		\itemМинимальный элемент каждой строки.
		\itemМаксимальный элемент каждой строки.
		\itemВыборочная дисперсия отрицательных элементов каждой строки.
		\itemМаксимальный среди отрицательных элементов каждой строки.
		\itemМатематическое ожидание отрицательных элементов.
	\end{enumerate}
	\parПараметр ''E'', определяющий метод сортировки вектора с результатами, принимает значения от одного до девяти, означающие:
	\begin{enumerate}
		\itemСортировка выбором (Selection sort).
		\itemСортировка Шелла (Shell sort) .
		\itemСортировка расчёской (Comb sort).
		\itemПирамидальная сортировка (сортировка кучи, Heapsort).
		\itemБыстрая сортировка (Quicksort).
		\itemПоразрядная сортировка (цифровая сортировка).
		\itemСортировка подсчётом (Counting sort).
		\itemГномья сортировка.
	\end{enumerate}
}