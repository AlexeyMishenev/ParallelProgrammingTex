{ %section9_1
	\subsection{Порядок выполнения работы}
	\begin{enumerate}
		\itemВам необходимо реализовать один (для оценки 3) или два (для оценки 4) этапа вашей программы из предыдущих лабораторных работ. При этом вычисления можно проводить как на CPU, так и на GPU (на своё усмотрение, но GPU предпочтительнее).
		\item\textbf{Дополнительной задание (оценка 5).}
			\begin{itemize}
				\itemВыполнение заданий для оценки 3 и 4.
				\itemРасчёт доверительного интервала. 
				\itemПосчитать время 2 способами: с помощью profiling и с помощью обычного замера (как в предыдущих заданиях).
				\itemОценить накладные расходы, такие как доля времени, проводимого на каждом этапе вычисления («нормированная
диаграмма с областями и накоплением»), число строк кода, добавленных при распараллеливании, а также грубая оценка времени, потраченного на распараллеливание (накладные расходы программиста), и т.п.
				\item* Необязательное задание для магистрантов с большим количеством свободного времени: проводить вычисления совместно на GPU и CPU (т.е. итерации в некоторой обоснованной пропорции делятся между GPU и CPU, и параллельно на них выполняются).
			\end{itemize}
		\itemПри желании данную лабораторную работу можно написать на CUDA.
	\end{enumerate}
}
