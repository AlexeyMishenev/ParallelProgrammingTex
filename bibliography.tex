{ %bibliography
	\phantomsection
	\section*{Список используемой литературы}
	\addcontentsline{toc}{section}{Список используемой литературы}
	\begin{enumerate}
		\sloppy
		\itemСоснин В.В., Балакшин П.В. Введение в параллельные вычисления. – СПб: Университет ИТМО, 2016. – 51 с.
		\item Top 500 supercomputers list. URL: \url{https://www.top500.org/} (дата обращения: 12.02.19).
		\itemВикипедия. Симметричная мультипроцессорность. URL: \url{https://ru.wikipedia.org/wiki/%D0%A1%D0%B8%D0%BC%D0%BC%D0%B5%D1%82%D1%80%D0%B8%D1%87%D0%BD%D0%B0%D1%8F_%D0%BC%D1%83%D0%BB%D1%8C%D1%82%D0%B8%D0%BF%D1%80%D0%BE%D1%86%D0%B5%D1%81%D1%81%D0%BE%D1%80%D0%BD%D0%BE%D1%81%D1%82%D1%8C} (дата обращения: 12.02.19).
		\itemВикипедия. Массово-параллельная архитектура. URL: \url{https://ru.wikipedia.org/wiki/%D0%9C%D0%B0%D1%81%D1%81%D0%BE%D0%B2%D0%BE-%D0%BF%D0%B0%D1%80%D0%B0%D0%BB%D0%BB%D0%B5%D0%BB%D1%8C%D0%BD%D0%B0%D1%8F_%D0%B0%D1%80%D1%85%D0%B8%D1%82%D0%B5%D0%BA%D1%82%D1%83%D1%80%D0%B0} (дата обращения: 12.02.19).
		\itemВикипедия. OpenCL. URL: \url{https://ru.wikipedia.org/wiki/OpenCL} (дата обращения: 13.02.19).
		\itemВведение в GPU-вычисления - CUDA/OpenCL. URL: \url{http://my-it-notes.com/2013/06/gpu-processing-intro-cuda-opencl/} (дата обращения: 13.02.19).
		\itemБастраков С.И. Программирование на OpenCL. - Нижний новгород: ННГУ, 2011.
		\item OpenCL – официальный сайт URL \url{:http://www.khronos.org/opencl/}
	\end{enumerate}
}