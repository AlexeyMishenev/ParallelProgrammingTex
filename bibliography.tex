{ %bibliography
	\phantomsection
	\section*{Список используемой литературы}
	\addcontentsline{toc}{section}{Список используемой литературы}
	\begin{enumerate}
		\sloppy
		\itemСоснин В.В., Балакшин П.В. Введение в параллельные вычисления. – СПб: Университет ИТМО, 2016. – 51 с.
		\item Top 500 supercomputers list. URL: \href{https://www.top500.org/} {https://www.top500.org/} (дата обращения: 12.02.19).
		\itemВикипедия. Симметричная мультипроцессорность. URL: \href{https://ru.wikipedia.org/wiki/%D0%A1%D0%B8%D0%BC%D0%BC%D0%B5%D1%82%D1%80%D0%B8%D1%87%D0%BD%D0%B0%D1%8F_%D0%BC%D0%BD%D0%BE%D0%B3%D0%BE%D0%BF%D1%80%D0%BE%D1%86%D0%B5%D1%81%D1%81%D0%BE%D1%80%D0%BD%D0%BE%D1%81%D1%82%D1%8C}{https://ru.wikipedia.org/wiki/Симметричная\_многопроцессорность} (дата обращения: 12.02.19).
		\itemВикипедия. Массово-параллельная архитектура. URL: \href{https://ru.wikipedia.org/wiki/%D0%9C%D0%B0%D1%81%D1%81%D0%BE%D0%B2%D0%BE-%D0%BF%D0%B0%D1%80%D0%B0%D0%BB%D0%BB%D0%B5%D0%BB%D1%8C%D0%BD%D0%B0%D1%8F_%D0%B0%D1%80%D1%85%D0%B8%D1%82%D0%B5%D0%BA%D1%82%D1%83%D1%80%D0%B0}{https://ru.wikipedia.org/wiki/Массово-параллельная архитектура} (дата обращения: 12.02.19).
		\itemВикипедия. OpenCL. URL: \href{https://ru.wikipedia.org/wiki/OpenCL}{https://ru.wikipedia.org/wiki/OpenCL} (дата обращения: 13.02.19).
		\itemВведение в GPU-вычисления - CUDA/OpenCL. URL: \href{http://my-it-notes.com/2013/06/gpu-processing-intro-cuda-opencl/}{http://my-it-notes.com/2013/06/gpu-processing-intro-cuda-opencl/} (дата обращения: 13.02.19).
		\itemБастраков С.И. Программирование на OpenCL. - Нижний новгород: ННГУ, 2011.
		\item OpenCL – официальный сайт URL: \href{http://www.khronos.org/opencl/}{http://www.khronos.org/opencl/} (дата обращения: 13.02.19).
		\itemАнтонов А.С. Параллельное программирование с использованием технологии MPI. - Москва: МГУ, 2004. - 72 с.
		\itemАнтонов А.С. Параллельное программирование с использованием технологии OpenMP. - Москва: МГУ, 2009.- 78 с.
		\itemВикипедия. Параллельные вычисления. URL: \href{https://ru.wikipedia.org/wiki/%D0%9F%D0%B0%D1%80%D0%B0%D0%BB%D0%BB%D0%B5%D0%BB%D1%8C%D0%BD%D1%8B%D0%B5_%D0%B2%D1%8B%D1%87%D0%B8%D1%81%D0%BB%D0%B5%D0%BD%D0%B8%D1%8F} {https://ru.wikipedia.org/wiki/Параллельные\_вычисления} (дата обращения: 14.02.19).
		\itemВикипедия. Распределенные вычисления. URL: \href{https://ru.wikipedia.org/wiki/%D0%A0%D0%B0%D1%81%D0%BF%D1%80%D0%B5%D0%B4%D0%B5%D0%BB%D1%91%D0%BD%D0%BD%D1%8B%D0%B5_%D0%B2%D1%8B%D1%87%D0%B8%D1%81%D0%BB%D0%B5%D0%BD%D0%B8%D1%8F} {https://ru.wikipedia.org/wiki/Распределенные\_вычисления}(дата обращения: 14.02.19).
		\itemСтандарт языка С++. ISO/IEC 14882:2011. URL: \href{https://www.iso.org/standard/50372.html}{https://www.iso.org/standard/50372.html} (дата обращения: 14.02.19).
		\itemВикипедия. Проблема АВА. URL: \href{https://ru.wikipedia.org/wiki/%D0%9F%D1%80%D0%BE%D0%B1%D0%BB%D0%B5%D0%BC%D0%B0_ABA} {https://ru.wikipedia.org/wiki/Проблема\_АВА} (дата обращения: 14.02.19).
		\itemТранзакционная память: история и развитие. URL: \href{https://habr.com/ru/post/221667/}{https://habr.com/ru/post/221667/} (дата обращения: 14.02.19).
		\itemВикипедия. Модель акторов. URL: \href{https://ru.wikipedia.org/wiki/%D0%9C%D0%BE%D0%B4%D0%B5%D0%BB%D1%8C_%D0%B0%D0%BA%D1%82%D0%BE%D1%80%D0%BE%D0%B2}{https://ru.wikipedia.org/wiki/Модель_Акторов} (дата обращения: 14.02.19).
		\itemКормен, Томас Х. и др. Алгоритмы: построение и анализ, 3-е изд. : Пер. с англ. - М. : ООО ''И. Д. Вильямс'', 2013. - 1328 с. : ил. - Парал. тит. англ.
		\item Lock-free структуры данных. Очередной трактат. URL: \href{https://habr.com/ru/post/219201/}{https://habr.com/ru/post/219201/} (дата обращения: 14.09.19)
		\item False sharing в многопоточном приложении на Java. URL: \href{https://habr.com/ru/post/187752/}{https://habr.com/ru/post/187752/} (дата обращения: 14.09.19)
		\item Load-link/store-conditional. URL: \href{https://en.wikipedia.org/wiki/Load-link/store-conditional}{https://en.wikipedia.org/wiki/Load-link/store-conditional} (дата обращения: 14.09.19)
		\item Concurrent Data Structures (libcds). URL: \href{http://libcds.sourceforge.net/}{http://libcds.sourceforge.net/} (дата обращения: 14.09.19)
		\item Очередь Майкла и Скотта. URL: \href{https://neerc.ifmo.ru/wiki/index.php?title=%D0%9E%D1%87%D0%B5%D1%80%D0%B5%D0%B4%D1%8C_%D0%9C%D0%B0%D0%B9%D0%BA%D0%BB%D0%B0_%D0%B8_%D0%A1%D0%BA%D0%BE%D1%82%D1%82%D0%B0} {https://neerc.ifmo.ru/wiki/index.php?title=Очередь\_Майкла\_и\_Скотта} (дата обращения: 14.09.19)
		\item Баер Ж.-Л., Барлоу Р., Вудворд М., и др. Системы параллельной обработки: Пер. с англ. / Под. ред. Д. Ивенса. - М.: Мир, 1985. - 416 с.: ил.
		\item Валях Е. Последовательно-параллельные вычисления. Пер. с англ. - М.: Мир, 1985. - 456 с.: ил.
		\item Воеводин В.В., Воеводин Вл.В. Параллельные вычисления. - СПб.: БХВ-Петербург, 2002. - 608 с.: ил.
	\end{enumerate}
}