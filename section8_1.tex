{ %section8_1
	\subsection{Порядок выполнения работы}
	\begin{enumerate}
		\itemВзять в качестве исходной OpenMP-программу из ЛР-5, в которой распараллелены все этапы вычисления. Убедиться, что в этой программе корректно реализован одновременный доступ к общей переменной, используемой для вывода в консоль процента завершения программы.
		\itemИзменить исходную программу так, чтобы вместо OpenMP-директив применялся стандарт «POSIX Threads»:
			\begin{itemize}
				\itemдля получения оценки \textbf{«3»} достаточно изменить только один этап (Generate, Map, Merge, Sort), который является узким местом (bottle neck), а также функцию вывода в консоль процента завершения программы;
				\itemдля получения оценки \textbf{«4»} необходимо изменить всю программу, но допускается в качестве расписания циклов использовать «schedule static»;
				\itemдля получения оценки \textbf{«5»} необходимо хотя бы один цикл распараллелить, реализовав вручную расписание «schedule dynamic» или «schedule guided».
			\end{itemize}
		\itemПровести эксперименты и по результатам выполнить сравнение работы двух параллельных программ («OpenMP» и «POSIX Threads»), которое должно описывать следующие аспекты работы обеих программ (для различных $N$):
			\begin{itemize}
				\itemполное время решения задачи;
				\itemпараллельное ускорение;
				\itemдоля времени, проводимого на каждом этапе вычисления («нормированная
диаграмма с областями и накоплением»);
				\itemколичество строк кода, добавленных при распараллеливании, а также грубая оценка
времени, потраченного на распараллеливание (накладные расходы программиста);
				\itemостальные аспекты, которые вы выяснили самостоятельно (\textbf{Обязательный пункт});
			\end{itemize}
	\end{enumerate}
}