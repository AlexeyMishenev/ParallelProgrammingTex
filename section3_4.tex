{ %section3_4
	\subsection{Ошибки в многопоточных приложениях}
	\Large\parПомимо привычных для программиста ошибок, встречающихся в компьютерных программах, существует ряд ошибок, специфичных для параллельного программирования. Эти ошибки обусловлены следующими особенностями параллельных программ: 
	\begin{itemize}
		\item\textbf{Синхронизация потоков.} Программист должен обеспечить корректную последовательность выполняемых разными потоками операций. В общем случае невозможно точно сказать, в какой последовательности будут выполняться команды потоков, т.к. операционная система может в произвольный момент времени при остановить выполнение потока.
		\item\textbf{Взаимодействие потоков.} Также программист не должен конфликтов при обращении к общим для потоков областям памяти. 
		\item\textbf{Балансировка нагрузки.} Если в распараллеленной программе один из потоков выполняет 99\% работы, то даже на 64-ядерной системе параллельное ускорение едва ли превысит значение 1.01.
		\item\textbf{Масштабируемость.} В идеале параллельная программа должна одинаково хорошо распараллеливать выполняемую работу на любом доступном количестве процессоров. Однако добиться этого нелегко и это часто приводит к трудно обнаруживаемым ошибкам.
	\end{itemize}
	\parРассмотрим далее подробнее следующий неполный перечень типовых ошибок, возникающих в параллельных программах независимо от используемой технологии распараллеливания:
	\begin{itemize}
		\itemПотеря точности операций с плавающей точкой.
		\itemВзаимные блокировки (deadlock).
		\itemСостояния гонки (race conditions). 
		\itemПроблема АВА.
		\itemИнверсия приоритетов.
	\end{itemize}
	\par\textbf{Потеря точности.} Если параллельная программа используется для проведения операций с плавающей точкой при работе с вещественными  переменными, расположенными в общей для потоков памяти, то при каждом запуске программы может получаться разный результат вещественных расчётов. Это объясняется тем, что при работе нескольких потоков невозможно точно предсказать, в каком порядке операционная система предоставит этим потокам процессор, т.к. в любой момент любой поток может быть временно приостановлен по усмотрению ОС. Это в свою очередь приводит к неопределённой последовательности выполнения операций с плавающей точкой, результат которых, как известно, может зависеть от порядка.
	\parРассмотрим пример, иллюстрирующий сказанное:
	
}